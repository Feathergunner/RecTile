\documentclass[a4paper,11pt]{article}
% the report class allows you to use chapters as a super-section as is the best option for a thesis
% each chapter will start on a new page
% for smaller documents, like handouts or a seminar work, the basic article is fine too

% LAYOUT & FORMATTING ----------------------------------
% adjust the geometry of the page, s.t. more text fits on less pages ;)
\usepackage{geometry}
\geometry{a4paper, top=25mm, left=35mm, right=20mm, bottom=25mm, headsep=10mm, footskip=12mm}
% general symbol encoding stuff, so you need less special commands for symbols that are already on the keyboard
\usepackage[utf8]{inputenc}
\usepackage[T1]{fontenc}

% hyperref is always nice to have (clickable links), but I don't like the colored boxes everywhere in the text
\usepackage[hidelinks]{hyperref}

% more option on image placement
\usepackage{float}

% enforces some specific way of optimizing whitespace (? not sure anymore what this does exactly)
\frenchspacing
% ------------------------------------------------------

% FANCY HEADER -----------------------------------------
\usepackage{fancyhdr}
% define custom pageheader and pagefoot
% plain: totally empty, used when no header is wanted
\fancypagestyle{plain}{
	\fancyhf{}
	\fancyhead[LE,RO]{\thepage}
	\fancyfoot[LE,RO]{\thepage}
}
\pagestyle{plain}
% ------------------------------------------------------

% TABLES OF CONTENTS -----------------------------------
% depth of table of contents
% (ie the max depth of the sub-sub-section-tree that is shown in the table of contents)
\setcounter{tocdepth}{1}

% basic style for biliography:
\bibliographystyle{plain}
% ------------------------------------------------------

% TABLES -----------------------------------------------
% stuff to optimize tables:
\usepackage[table, dvipsnames]{xcolor}
% longtables can span multiple pages, and they have their header repeated each page
\usepackage{longtable}
\usepackage{tabularx}
\usepackage{colortbl}
% ------------------------------------------------------

% MATHTOOLS --------------------------------------------
% commands to display formulas and a large variety of mathematical symbols
\usepackage{bbm}
\usepackage[fleqn]{amsmath}
\usepackage{amssymb}
\usepackage{amsthm}
\usepackage{thmtools}

\usepackage{mathabx}
% ------------------------------------------------------

% AMSTHM -----------------------------------------------
% a very plain style for definitions, theorems etc.
\newtheoremstyle{mystyle}% name
  {}%         Space above, empty = `usual value'
  {}%         Space below
  {\itshape}% Body font
  {}%         Indent amount (empty = no indent, \parindent = para indent)
  {\bfseries}% Thm head font
  {.}%        Punctuation after thm head
  {\newline}% Space after thm head: \newline = linebreak
  {}%         Thm head spec

% some custom 'theorem'-like environments that each get their own counter  
\theoremstyle{mystyle}
\newtheorem{exmp}{Example}
\newtheorem{theo}{Theorem}
\newtheorem{lemm}{Lemma}
\newtheorem{defn}{Definition}
\newtheorem{prob}{Problem}
\newtheorem{algo}{Algorithm}
\newtheorem{ass}{Assumption}
\newtheorem{obs}{Observation}
% ------------------------------------------------------

% IMAGES AND PLOTS -------------------------------------
% to display images and self-defined graphs
\usepackage{graphicx}
\usepackage{tikz}
\usetikzlibrary{patterns, snakes, shapes.misc, positioning}
% with subcaption, one can use subfigures, i.e. "Figure 2a"
\usepackage{subcaption}
% ------------------------------------------------------

% CODE AND PSEUDO-CODE ---------------------------------
% algorithm2e is a nice package for peudo-code
\usepackage[ruled,linesnumbered,dotocloa]{algorithm2e}
% 'real' code can be displayed with listing
\usepackage{listings}
% ------------------------------------------------------

% COMMANDS ---------------------------------------------
% custom commands, so one does not have to type the whole word every time :)
\newcommand{\dref}[1]{Definition~\ref{#1}}
\newcommand{\lref}[1]{Lemma~\ref{#1}}
\newcommand{\eref}[1]{Equation~\ref{#1}}
\newcommand{\exref}[1]{Example~\ref{#1}}
\newcommand{\fref}[1]{Figure~\ref{#1}}
\newcommand{\sref}[1]{Section~\ref{#1}}
\newcommand{\cref}[1]{Chapter~\ref{#1}}
\newcommand{\aref}[1]{Algorithm~\ref{#1}}
\newcommand{\tref}[1]{Table~\ref{#1}}
\newcommand{\appref}[1]{Appendix~\ref{#1}}

% shorthands for common useds stuff thats difficult to type
\newcommand{\shortwlog}{w.\,l.\,o.\,g.\ }
\newcommand{\ie}{i.\,e.\ }
\newcommand{\st}{s.\,t.\ }
\newcommand{\eg}{e.\,g.\ }
\newcommand{\iid}{i.\,i.\,d.\ }
\newcommand{\etal}{et al.\ }

\newcommand{\transp}{^\mathsf{T}}

% META COMMANDS ----------------------------------------
\newcommand{\cn}{\footnote{Citation Needed}}
% ------------------------------------------------------

\title{RecTiles}
\author{Andreas Goral}
\begin{document}
%\begin{titlepage}
\thispagestyle{empty}
\begin{center}
\textsc{
{\huge Recursive Tiling}}\\[1cm]
%\vfill
%Copyright (c) 2020 Andreas Goral
\end{center}
%\end{titlepage}
\tableofcontents
\newpage
\section{Basics}
Aim of this document is to describe recursive algorithms to construct arbitrary big matrices $A^\star \in \{0,\dots, k^\star\}^{n^\star \times m^\star}$ from one (or more) significantly smaller matrices $A_0, \dots, A_\ell$ with $A_i \in \{0, \dots ,k_i\}^{n_i \times m_i}$.

In general, we will have $k_i = k_j \leq k^\star$, $n_i = n_j$ and $m_i = m_j ~ \forall i, j$. 

After constructing $A^\star$, we use a mapping from $\{0, \dots, k^\star\}$ to a set of colors and save the matrix as an image.

\section{Notation}
\begin{itemize}
	\item As noted above, $A_0, \dots, A_i, \dots, A_\ell$ are the base matrices that are the input to our algorithms. $n_i$ and $m_i$ are the corresponding dimensions of matrix $A_i$.
	\item In constrast, $A^{(i)}$ will describe the matrix we have at the $i$-th iteration of the algorithm. Consequently, $n^{(i)}$ and $m^{(i)}$ are the dimensions of $A^{(i)}$.
	\item In order to not confuse the indices of a matrix from a set of matrices (\eg $A_i$ from $A_0, \dots, A_k$) with the indices of the elements of $A_i$, we will sometimes use the programming-language-inspired notation of $A_i\left[x,y\right]$ to identify the element at position $x,y$ of matrix $A_i$.
\end{itemize}

\section{Recursive Tiling}
This approach is inspired by \cite{BaakeG13} (see Chapter 6, Figure 6.25, it's called \emph{block inflation} there).
The basic idea is as follows: we have the slightly stronger requirements $n := n_i = n_j , m := m_i=m_j \forall i,j$ and $\ell = k_i = k^\star ~ \forall i$, \ie all base matrices $A_i$ have the same size and there are exactly as many base matrices as there are possible different matrix elements.

\subsection{A recursive tiling step using inflation tiling}
We start with an arbitrary matrix $A^{(0)} \in \{0, \dots, k^\star\}^{n_0 \times m_0}$ (\eg we can use $A^{(0)} = A_0$).
Then, in iteration $i$, $A^{(i+1)}$ is constructed from $A^{(i)}$ by repacing each entry $a_{x,y}$ of $A^{(i)}$ by the base matrix $A_{a_{x,y}}$.
Thus, $A^{(i+1)}$ is a block matrix of $n^{(i)} \times m^{(i)}$ many blocks $B_{x,y}$ of size $n \times m$ each that is conform to the following simple rule:
\begin{align*}
	B_{x,y} = A_j \Leftrightarrow A^{(i)}_{x,y} = j.
\end{align*}
%for each entry in $A^{(i)}$.% Therefore, if $A^{(i)}$ has size $n^{(i)}} \times m^{(i)}$

\begin{exmp}\label{ex1}
Let
\begin{align*}
A_0 = \begin{pmatrix}
1 & 0\\
0 & 1
\end{pmatrix},\\
A_1 = \begin{pmatrix}
0 & 1\\
1 & 0
\end{pmatrix}
\end{align*}
and $A^{(0)} = A_0$. Then after one iteration, we have
\begin{align*}
A^{(1)} & = \begin{pmatrix}
A_1 & \vline & A_0\\ \hline
A_0 & \vline & A_1
\end{pmatrix}\\
& = \begin{pmatrix}
\begin{matrix}
0 & 1\\
1 & 0
\end{matrix} & \vline &
\begin{matrix}
1 & 0 \\
0 & 1
\end{matrix} \\ \hline
\begin{matrix}
1 & 0\\
0 & 1
\end{matrix} & \vline &
\begin{matrix}
0 & 1 \\
1 & 0
\end{matrix}
\end{pmatrix}.
\end{align*}
After the second iteration, we have
\begin{align*}
A^{(2)} & = \begin{pmatrix}
A_0 & \vline & A_1 & \vline & A_1 & \vline & A_0 \\ \hline
A_1 & \vline & A_0 & \vline & A_0 & \vline & A_1 \\ \hline
A_1 & \vline & A_0 & \vline & A_0 & \vline & A_1 \\ \hline
A_0 & \vline & A_1 & \vline & A_1 & \vline & A_0
\end{pmatrix} \\
& = \begin{pmatrix}
\begin{matrix}
1 & 0\\
0 & 1
\end{matrix} & \vline &
\begin{matrix}
0 & 1\\
1 & 0
\end{matrix} & \vline &
\begin{matrix}
0 & 1\\
1 & 0
\end{matrix} & \vline &
\begin{matrix}
1 & 0\\
0 & 1
\end{matrix} \\ \hline
\begin{matrix}
0 & 1\\
1 & 0
\end{matrix} & \vline &
\begin{matrix}
1 & 0\\
0 & 1
\end{matrix} & \vline &
\begin{matrix}
1 & 0\\
0 & 1
\end{matrix} & \vline &
\begin{matrix}
0 & 1\\
1 & 0
\end{matrix}\\ \hline
\begin{matrix}
0 & 1\\
1 & 0
\end{matrix} & \vline &
\begin{matrix}
1 & 0\\
0 & 1
\end{matrix} & \vline &
\begin{matrix}
1 & 0\\
0 & 1
\end{matrix} & \vline &
\begin{matrix}
0 & 1\\
1 & 0
\end{matrix}\\ \hline
\begin{matrix}
1 & 0\\
0 & 1
\end{matrix} & \vline &
\begin{matrix}
0 & 1\\
1 & 0
\end{matrix} & \vline &
\begin{matrix}
0 & 1\\
1 & 0
\end{matrix} & \vline &
\begin{matrix}
1 & 0\\
0 & 1
\end{matrix}
\end{pmatrix}
\end{align*}
\end{exmp}

\section{Outer Sum}
This time, the requirements are a little less restricted. We can use a set of arbitrary many matrices $A_1, \dots, A_\ell$ that can all have arbitrary sizes and elements, \ie $A_i \in \{0, \dots, k_i\}^{n_i \times m_i}$ (However, $k_i>k^\star$ does not make much sense for the algorithm).

\subsection{Preliminaries}
In the spirit of the Kronecker Product, we define an ``\emph{outer sum}''¸ of matrices as follows: Let $A_1 \in \mathbb{R}^{n_1 \times m_1}, A_2 \in \mathbb{R}^{n_2 \times m_2}$ be two matrices. The \emph{outer sum} of $A_1$ and $A_2$ is a block matrix $B = A_1 \boxplus A_2 \in \mathbb{R}^{n_1 \cdot n_2 \times m_1 \cdot m_2}$ that consits of $n_2 \times m_2$ many blocks of size $n_1 \times m_1$.

The elements of the block $B_{i_x,i_y}$ are
\begin{align*}
B_{i_x,i_y}\left[j_x, j_y\right] = A_1\left[j_x, j_y \right] + A_2\left[ i_x, i_y\right].
\end{align*}
That is, the full matrix looks as follows:
{\tiny
\begin{align*}
& B = A_1 \boxplus A_2 =\\
& \begin{pmatrix}
\begin{matrix}
A_1\left[1,1\right] + A_2\left[1,1\right] & \dots & A_1\left[1,m_1\right] + A_2\left[1,1\right] \\
\vdots & \ddots & \vdots \\
A_1\left[n_1,1\right] + A_2\left[1,1\right] & \dots & A_1\left[n_1,m_1\right] + A_2\left[1,1\right]
\end{matrix} & \vline & \dots & \vline &
\begin{matrix}
A_1\left[1,1\right] + A_2\left[1,m_2\right] & \dots & A_1\left[1,m_1\right] + A_2\left[1,m_2\right] \\
\vdots & \ddots & \vdots \\
A_1\left[n_1,1\right] + A_2\left[1,m_2\right] & \dots & A_1\left[n_1,m_1\right] + A_2\left[1,m_2\right]
\end{matrix}\\ \hline
\vdots & \vline & \ddots & \vline & \vdots \\ \hline
\begin{matrix}
A_1\left[1,1\right] + A_2\left[n_2,1\right] & \dots & A_1\left[1,m_1\right] + A_2\left[n_2,1\right] \\
\vdots & \ddots & \vdots \\
A_1\left[n_1,1\right] + A_2\left[n_2,1\right] & \dots & A_1\left[n_1,m_1\right] + A_2\left[n_2,1\right]
\end{matrix} & \vline & \dots & \vline &
\begin{matrix}
A_1\left[1,1\right] + A_2\left[n_2,m_2\right] & \dots & A_1\left[1,m_1\right] + A_2\left[n_2,m_2\right] \\
\vdots & \ddots & \vdots \\
A_1\left[n_1,1\right] + A_2\left[n_2,m_2\right] & \dots & A_1\left[n_1,m_1\right] + A_2\left[n_2,m_2\right]
\end{matrix}
\end{pmatrix}
\end{align*}
}


\subsection{A recursive tiling step using the outer sum}
This time, we have to start with  $A^{(0)}=A_0$ % a zero matrix of size $1 \times 1$, \ie $A^{(0)} = (0)$ 
(alternatively, we could start with any arbitrary matrix of arbitrary size, this would add a global offset to the result).

In iteration $i$ we compute $A^{(i)}$ as the outer sum of $A^{(i-1)}$ and $A_{i \text{mod} \ell}$. 
Note that due to the summing of elements, the absolute values of the matrix elements of $B$ might exceed our maximum allowed absolute value $k^\star$.
Therefore we apply the modulo-operation $\text{mod } (k^\star +1)$ on each element of $A^{(i)}$ afterwards. Thus, the full iteration step is

\begin{align*}
A^{(i)} = (A^{(i-1)} \boxplus A_{i \text{ mod } \ell}) \text{ mod } (k^\star +1).
\end{align*}

Note that is sufficient to apply the outer modulo only once after the last iteration.
%Note that $\boxplus$ is \emph{not commutative}.

\begin{exmp}
We will use the same matrices from \exref{ex1}, that is 
\begin{align*}
A_0 = \begin{pmatrix}
1 & 0\\
0 & 1
\end{pmatrix},\\
A_1 = \begin{pmatrix}
0 & 1\\
1 & 0
\end{pmatrix}
\end{align*}
and $A^{(0)} = A_0$. With $k^\star=2$, we have
\begin{align*}
A^{(1)} & = (A_0 \boxplus A_1) \text{ mod }k^\star \\
&= \left(\begin{pmatrix}
1 & 0\\
0 & 1
\end{pmatrix} \boxplus \begin{pmatrix}
0 & 1\\
1 & 0
\end{pmatrix}\right) \text{ mod }2 \\
& = \begin{pmatrix}
\begin{pmatrix}
1 & 0\\
0 & 1
\end{pmatrix} + 0 & \vline & \begin{pmatrix}
1 & 0\\
0 & 1
\end{pmatrix} + 1 \\ \hline
\begin{pmatrix}
1 & 0\\
0 & 1
\end{pmatrix} + 1 & \vline & \begin{pmatrix}
1 & 0\\
0 & 1
\end{pmatrix} + 0
\end{pmatrix} \text{ mod }2 \\
& = \begin{pmatrix}
\begin{matrix}
1 & 0\\
0 & 1
\end{matrix} & \vline &
\begin{matrix}
2 & 1\\
1 & 2
\end{matrix} \\ \hline
\begin{matrix}
2 & 1\\
1 & 2
\end{matrix} & \vline &
\begin{matrix}
1 & 0\\
0 & 1
\end{matrix}
\end{pmatrix} \text{ mod }2 \\
& = \begin{pmatrix}
\begin{matrix}
1 & 0\\
0 & 1
\end{matrix} & \vline &
\begin{matrix}
0 & 1\\
1 & 0
\end{matrix} \\ \hline
\begin{matrix}
0 & 1\\
1 & 0
\end{matrix} & \vline &
\begin{matrix}
1 & 0\\
0 & 1
\end{matrix}
\end{pmatrix}
\end{align*}

Note that we are free to choose a larger $k^\star$, this would yield a different result. The effect of $k^\star$ is increasing with each iteration.
\end{exmp}


\newpage
\bibliography{tiling}

\end{document}